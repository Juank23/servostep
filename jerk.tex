\documentclass{article}
\usepackage{amsmath}
\begin{document}

Jerk is the derivative of acceleration, or the third derivative of position.
We want to implement a {\em constant jerk} control law for the servo
controller.  We have three parameters as input: the target position $X$,
the maximum jerk $J$, the maximum acceleration $A$ and the maximum
velocity $V$.  Note that the maximums are limits; we do not want to
decellerate any faster than we accelerate.

First, the basics equations of motion:

\newcommand{\half}{\frac{1}{2}}

$$j(t) \in \{ -J, 0, +J \}$$
$$a(t) = \int j(t) dt = a_0 + j t$$
$$v(t) = \int a(t) dt = v_0 + a_0 t + \half j t^2$$
$$x(t) = \int v(t) dt = x_0 + v_0 t + \half a_0 t^2 + \frac{1}{6} j t^3$$

Assuming that the motor starts at rest, $a_0 = 0$, $v_0 = 0$ and $x_0 = 0$.
Initially we want to begin acceleration with $j(0) = +J$ until we reach
our maximum allowed acceleration $A$.

$$a(t_1) = A = a(t_0) + J t_1 \rightarrow t_1 = \frac{A}{J}$$

Our velocity at this point $t_1$ is:

\begin{align}
v(t_1) &= v(t_0) + a(t_0) t_1 + \half J t_1^2 \\
       &= \half J (\frac{A}{J})^2 \\
       &= \half \frac{A^2}{J}
\end{align}

At this point the jerk goes to zero and we continue to accelerate
at a constant acceleration. But for how long? We want to smoothly
decellerate so that the acceleration goes to zero exactly as we
achieve the desired velocity $V$.  By symmetry we can see that the
decelleration period is the same length as the acceleration period,
so we will gain the same amount of velocity as we did at $v(t_1)$.

\begin{align}
V &= 2 * v(t_1) + A t_2 \\
  &= 2 * \half \frac{A^2}{J} + A t_2 \\
  &= \frac{A^2}{J} + A t_2 \\
A t_2 &= V - \frac{A^2}{J} \\
  t_2 &= \frac{V}{A} - \frac{A}{J} \\
\end{align}

Therefor the time that we achieve our maximum velocity is

\begin{align}
t_v &= 2 t_1 + t_2 \\
    &= 2 \frac{A}{J} + \frac{V}{A} - \frac{A}{J} \\
    &= \frac{A}{J} + \frac{V}{A}
\end{align}

We could compute the distance 
And we have travelled:

\begin{align}
x(t_1) &= x(t_0) + v(t_0) t_1 + \half a(t_0) t_1^2 + \frac{1}{6} J t_1^3 \\
       &= \frac{1}{6} J (\frac{A}{J})^3 \\
       &= \frac{1}{6} \frac{A^3}{J^2}
\end{align}

During this time of constant acceleration we move an additional

\begin{align}
x(t_2) &= x(t_1) + v(t_1) t_2 + \half A t_2^2 \\
       &= \frac{1}{6} \frac{A^3}{J^2} + \half \frac{A^2}{J} (\frac{V}{A} - \frac{A}{J}) + \half A (\frac{V}{A} - \frac{A}{J})^2 \\
\end{align}

\end{document}
